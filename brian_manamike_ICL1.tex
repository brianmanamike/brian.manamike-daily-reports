\documentclass{article}
\usepackage{graphicx} % Required for inserting images
\usepackage{fancyvrb}
\usepackage{graphicx}
\usepackage{algorithm}
\usepackage{verbatim}

\title{AFRICAN INSTITUTE OF MATHEMATICS AND SCIENCES}
\author{BRIAN AYANDA MANAMIKE}

\date{}


\usepackage{graphicx}

\begin{document}

\maketitle
\section*{Question 1}
\begin{verbatim}
sumation(){
read -p "If you want to compute odd press 1 and for even press 2 : " a


#1(i)
if (( a==1 )) ;then
read -p "enter the number n:" n
sum=0
for(( i=1; i<=n; i+=2 ));
do
 sum=$(( sum + i + 3 ))
done
echo "$sum"

elif (( a==2 )) ;then
#1(ii)

read -p " enter number n:" n
sum=0
for(( i=1; i<=n; i++ ));
do
 if (( i%2==0 )) ;then
 sum=$(( sum + i + 3 ))

fi
done
echo "$sum"
fi

}

sumation
\end{verbatim}
\section*{Question 2}
\begin{verbatim}
Days_of_the week(){
read -p "Enter the number of a day you want to check : " n


my_arr=("Sunday"  "Monday" "Tuesday" "Wednesday" "Thursday" "Friday" "Saturday")
if [[ $n -gt 0 && $n -lt 6 ]] ;then
echo "The day is a working day and the day is ${my_arr[$n]} "
else
echo "The day is a weekend"
fi

}
Days_of_the_week
\end{verbatim}
\section*{Number 3}
\begin{verbatim}
get_number_of_months() {

read -p "Enter the number of the month :" n

my_arr=("January" "February" "March" "April" "May" "June" "July" "August" "September" "October" "November" "December")
if [[ $n -eq 1 ]] ;then
echo "The month is ${my_arr[$n]} and has 28 days"
elif [[ $n -eq 3 || $n -eq 5 || $n -eq 8 || $n -eq 10 ]] ;then
echo "The month is ${my_arr[$n]} and has 30 days"
else
echo "The month is ${my_arr[$n]} and has 31 days"
fi
}
get_number_of_months
\end{verbatim}
\section*{Number 4}
\begin{verbatim}
Display_name(){

read -p "Enter your name :" name

for i in {1..3}
do
 echo "$name"
done
}
Display_name
#4b
Display_name(){
read -p "Enter your name :"
count=1
max_count=3

while [ $count -le $max_count ]
do

echo "$name"
((count++))
done
}
Display_name
\end{verbatim}
\section*{Number 5}
\begin{verbatim}
Fibonacci_sequence() {

read -p "Enter the first number :" a
read -p "Enter the second number :" b
echo "the fibonacci sequence until 15 :"

for (( i=0; i<15 ; i++ ))
do

echo  "$a "
c=$((a + b))
a=$b
b=$c
done
}
Fibonacci_sequence
\end{verbatim}
\section*{Number 6}
\begin{verbatim}
Sequence_numbers(){
read -p "Enter your first number :" x0
read -p "Enter your second number :" x1

x[0]=$x0
x[1]=$x1
echo "the sequence x[i+2]= 1/2x[i+1] -  3x[i] until 30 :"


for (( i=2; i<30; i++ )) do


x[i]=$(echo "scale=4; 0.5 * ${x[$((i-1))]} - 3 *${x[$((i-2))]}" | bc)
echo "x[$i]=${x[i]}"

done

}

Sequence_numbers $x0 $x1

\end{verbatim}

\section*{Number 7a.i}

\begin{verbatim}
sequence_numbers() {
read -p "Enter the number of M :" M

sum=0
n=0
for (( i = 1; ;i++ )) ; do
sum=$(( sum+i*i ))
if [[ $sum -ge $M ]];then
    n=$i
break
fi
done
echo " $M has reached or exceeded by the square sum up to : $n"
}
sequence_numbers

\end{verbatim}

\section*{Number 7a.ii}

\begin{verbatim}
multiplication(){

read -p "Enter a number between 1 to 9  : " n
read -p "Enter the number where you want the multiplication to stop :  " N

for i in $(seq 1 $N); do
  echo "$n * $i" "=" "$(($n * $i))"
done
}

multiplication
\end{verbatim}
\section*{Number 7b.i}
\begin{verbatim}
read -p "Enter a natural number M: " M


sum=0

# Using a for loop
for (( n=1; ; n++ ))
do
    sum=$((sum + n * n))  # Add n^2 to the sum
    if (( sum >= M )); then
        echo "The smallest n such that 1^2 + 2^2 + ... + n^2 >= $M is: $n"
        break
    fi
done
\end{verbatim}

\section*{Number 7b.ii.}
\begin{verbatim}

read -p "Enter a natural number M: " M

n=1
sum=0\begin{verbatim}


while (( sum < M ))
do
    sum=$((sum + n * n))  # Add n^2 to the sum
    n=$((n + 1))  # Increment n
done

n=$((n - 1))

echo "The smallest n such that 1^2 + 2^2 + ... + n^2 >= $M is: $n"
\end{verbatim}

\section*{Number 8}

\begin{verbatim}
multiplication(){

read -p "Enter a number between 1 to 9  : " n
read -p "Enter the number where you want the multiplication to stop :  " N

for i in $(seq 1 $N); do
  echo "$n * $i" "=" "$(($n * $i))"
done
}
multiplication
\end{verbatim}
\section*{Number 9}
\begin{verbatim}
    

multiples() {

for i in $(seq 0 40); do
    #if (( i % 3 == 0 || i % 7 == 0 || i % 11 == 0 )) ;then
      #echo "$i"
    #fi
done

\end{verbatim}
\section*{Number 10}

\begin{verbatim}

get_input() {
    local prompt=$1
    local min=$2
    local max=$3
    local num

    while :; do
        read -p "$prompt" num
        if [[ "$num" =~ ^[0-9]+$ ]] && (( num >= min && num <= max )); then
            break
        else
            echo "Please enter a valid number between $min and $max."
        fi
    done
    echo $num
}

while :; do
    
    x=$(get_input "Enter an integer x (0-9): " 0 9)

    
    max=$(get_input "Enter an integer max (0-255): " 0 255)

    
    echo "Multiples of $x between 0 and $max are:"
    for (( i=0; i<=max; i+=x )); do
        if (( i > 0 )); then
            echo "$i"
        fi
    done

    read -p "Do you want to continue? (y/n): " answer
    if [[ "$answer" != "y" ]]; then
        echo "Goodbye!"
        break
    fi
done
\end{verbatim}
\end{document}
